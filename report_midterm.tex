\documentclass[11pt,a4paper,oneside]{report}
\usepackage{graphicx}
\usepackage{caption}
\usepackage{subcaption}

\begin{document}
\title{HPSC Assignment 4}
\author{Gregory Petropoulos}
\date{October 17, 2012}
\maketitle

\section{Implementation}
To implement Conway's game of life I gave each process two one dimensional array, field\_a and field\_b. 
I navigated through these arrays using offests.  
This array had the data decomposition mandated by the users choice of partition, checkerboard or slice, as well as a perimiter.  
This amounted to declaring array sizes with heights and widths that were two spaces bigger than the local heigh and local width of the relevant data from the file. 
Empty enteries were filled as follows:  any edge of the board was filled with zeros and any border shared with another process was filled with the appropriate ghost data from its neighbor(s).  
This method of filling made communication to update the ghosts easy and made the update method universal for looping over the interior data.  
To communicate the ghost rows I used point to point communication between pairs of processors.  
I arraganged this communication so that half of the processors would send first and the other half would recieve first; as a result my code does not rely on MPI buffering.
For the ghost rows to communicate correctly for a general slice data decomposition and a general checkerboard data decomposition I used a series of if conditions involving the processor number, the size of the board, and the number of rows and columns in the global partitioning scheme.  
Additionally, I used two stage forwarding to share ghost columns and then ghost rows avoiding the need to communicate diagnoals seperately.  

To avoid executing large memory copies regularly I alternate between using field\_a and field\_b as follows:

\begin{enumerate}
  \item use field\_a to update field\_b
  \item communicate field\_b ghosts
  \item use field\_b to update field\_a
  \item communicate field\_a ghosts
  \item use field\_a to update field\_b
  \item \dots
\end{enumerate}

Thus the iteration number being odd or even has consequences for which array I perform my ghost update and perform bug counts.  
I handle this by passing the iteration number to these subroutines and using a tertiary operator on the mod of the iteration number to choose which field is the relevant one.
Finally my program handles all the errors I have encountered grancefully.
If the user does not want to run measurmetns they can enter an argument of zero.
Simply entering an argument larger than the number of iterations will result in the read in file being measured.

\section{Code}
To teset the code I ran the required tests from the assignment sheet shown in table \ref{runs}.
Each test ran sucesfully and generated the same output.  
The output is attached to this report.
\begin{table}
  \centering
  \begin{tabular}{|c|c|c|c|}
    \hline
    Nodes & Serial & Slice & Checkerboard \\
    \hline
        1 & run & & \\
        4 & & run & run \\
        9 & & run & run \\
       16 & & & run \\
       25 & & run & run \\
       36 & & run & run \\
    \hline
  \end{tabular}
  \caption{`run' indicates which tests were ran.}
  \label{runs}
\end{table}
\end{document}
